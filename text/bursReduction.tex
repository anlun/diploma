\subsection{Сведение Pretty-Printing к BURS}
К сожалению, библиотека, описанная в \cite{swierstra}, обладает экспоненциальной
сложностью относительно входного документа. Для снижения алгоритмической сложности
в рамках данной работы была произведено сведение задачи к BURS.

Сведение основано на следующих наблюдениях. Пусть $w$ есть максимальная допустимая
ширина вывода. Поскольку каждая возникающая раскладка представляется блоком
\lstinline[language=Haskell]{Format}, для поиска оптимального представления можно рассматривать
все промежуточные раскладки
как тройки $(n, k, h)$, где $n \le w$ --- это общая ширина раскладки, $k \le n$ ---
ширина ее последней строчки, $h$ --- ее высота. Тогда очевидно, что при $h_1 \le h_2$
между $(n, k, h_1)$ и $(n, k, h_2)$ первая тройка предпочтительней на любом этапе вычислений.
Таким образом для каждого узла не может быть более $w^2$ существенных представлений\footnote{
На самом деле, из-за ограничения $k \le n$ максимальное число раскладок не $w^2$, a
$\frac{w^2 + w}{2}$, но это несущественно.}.

Документ, для которого мы ищем раскладку, можно рассматривать как дерево, построенное из примитивов
\lstinline[language=Haskell]{text},
\lstinline[language=Haskell]{indent},
\lstinline[language=Haskell]{beside},
\lstinline[language=Haskell]{above} и \lstinline[language=Haskell]{choice}. Тогда задачу
раскладки документа можно решать независимо для поддеревьев, а потом объединять решения
при учете, что для каждого поддерева
и каждой пары $(n, k), k \le n \le w$ запоминается минимальная такая $h$, что для поддерева существует
текстовое представление с размерами $(n, k, h)$. Тогда, совершив обход дерева снизу вверх, мы сможем
вычислить оптимальное представление для всего дерева. Полезно заметить,
что оптимальное представление дерева не всегда получается из оптимальных представлений его
поддеревьев.

Эти наблюдения можно формализовать в виде BURS-задачи.
Для заданной ширины вывода $w$ введем набор нетерминалов
$T_n^k$, для всех $k \le n \le w$. Определим BURS грамматику так, чтобы цена вывода $h$ нетерминала
$T_n^k$ для документа соответствовала его раскладке с параметрами $(n, k, h)$.
Для такой BURS грамматики этап разметки посчитает все полезные раскладки, а этап свертки
вернет оптимальную. Определим правила переписывания для этой грамматики:
\begin{enumerate}
\item Для терминального узла $[\mbox{\lstinline{text s}}]$\footnote{
  Квадратные скобки используются для обозначения терминалов, состоящих из нескольких символов.}
  существует два варианта:
  \begin{itemize}
     \item Если $|s|\le w$ (где $|s|$ --- это длина строки $s$), то вводится единственное правило
           $T^{|s|}_{|s|}: [\mbox{\lstinline{text s}}]$ с ценой $1$;
           
           для всех остальных
           $k, n\ne |s|$ используется $T^k_n:[\mbox{\lstinline{text s}}]$ с ценой $\infty$;
     \item Если $|s| > w$, то используется $T^k_n:[\mbox{\lstinline{text s}}]$ с ценой\
       $\infty$ для всех $k, n$.
  \end{itemize}
  Действительно, раскладка документа, состоящего из строчки длины $|s|$, может быть только размеров
  $(|s|, |s|, 1)$. Все остальные размеры недоступны, поэтому имеют цену $\infty$.

\item Для узла $[\mbox{\lstinline{indent m}}]$ введем два набора правил:
  \begin{enumerate}
     \item $T^{k+m}_{n+m}:[\mbox{\lstinline{indent m}}]\:(T^k_n)$ с ценой,
       равной цене раскладки поддерева $m$ в $T_n^k$ , для всех $n$ 
     и $k$ таких, что $n+m\le w$ и $k\le n$;
     \item $T^k_n:[\mbox{\lstinline{indent m}}]\:(T^i_j)$ с ценой $\infty$ в противном случае.
  \end{enumerate}
  Понятно, что сдвиг раскладки с параметрами $n$, $k$ и $h$ на $m$ позиций вправо
  создает раскладку с параметрами $n+m$, $k+m$, $h$. Такая раскладка допустима, если
  $n+m\le w$ и $k+m\le w$.

\item Для узла $[\mbox{\lstinline{above}}]$ вводим правило $T^{k_2}_{\max(n_1, n_2)}:[\mbox{\lstinline{above}}]\:(T^{k_1}_{n_1},\;T^{k_2}_{n_2})$ 
с ценой, равной сумме цен вывода поддеревьев, для всех $k_1\le n_1\le w$ и $k_2 \le n_2 \le w$.

Действительно, при вертикальном соединении раскладок с параметрами
$n_1$, $k_1$, $h_1$ и $n_2$, $k_2$, $h_2$ мы получаем раскладку с размерами
$\max\:(n_1,n_2)$, $k_2$, $h_1+h_2$. Вертикальная композиция допустимых раскладок всегда допустима.

\item Для узла $[\mbox{\lstinline{beside}}]$ вводим правило
  $T^{k_1+k_2}_{\max\:(n_1,\:k_1+n_2)}:[\mbox{\lstinline{beside}}]\:(T^{k_1}_{n_1},\:T^{k_2}_{n_2})$
  для каждой комбинации $n_1, n_2, k_1, k_2$ такой, что $k_1+k_2\le\max\:(n_1,\:k_1+n_2)\le w$.
  Цена такого вывода равна сумме цен вывод поддеревьев минус 1.
  Может быть легко проверена геометрическими соображениями.

\item Для узла $[\mbox{\lstinline{choice}}]$ вводим правило
  $T^k_n:[\mbox{\lstinline{choice}}]\:(T^k_n,\:T^k_n)$ для всех $k\le n\le w$.

  Цена есть минимум среди цен вывода для поддеревьев. Понятно, что среди двух раскладок
  с одинаковыми ширинами выбирается та, что имеет меньшую высоту.
\end{enumerate}

Для завершения описания грамматики необходимо ввести правило для стартового нетерминала $S$.
Можно добавить правило $S : r$ с тождественной функцией цены для любого $r$, или ввести
цепное праивло $S: T_n^k$ с такой же функцией для всех нетерминалов $T_n^k$
(такое правило требует несущественного расширения определения BURS).

Число нетерминалов в определенной грамматике есть $O(w^2)$. Однако, число правил есть $O(w^4)$, так
как в дереве есть узлы со степенью 2. Так, приведенная BURS реализация оптимального принтера
работает за линейное время от числа узлов в дереве документа для фиксированной ширины вывода, при
этом сложность от ширины вывода есть $O(w^4)$. Понятно, что данное сведение может быть выполнено и
для других примитивов построения дерева, которые могут иметь большую степень, но ценой
экспоненциального роста сложности по $w$.
