\section{Обзор}

\subsection{Принтер-комбинаторы}

Простые, по завету Oppen. Есть линейная реализация, и реализация On-line и
много чего еще, но слабая выразительная мощь.

\subsection{Принтер-комбинаторы с выбором}
Достаточная выразительная сила, но существующая реализация экспоненциальна
относительно размеров входа.

\subsubsection{Структура Format}

\subsection{Средства форматирования кода в IDE}

\subsubsection{IDEA}
adas
\subsubsection{Eclipse}
sdasd

\subsection{BURS}

Bottom-Up Rewrite System (BURS)\cite{burs} --- это метод динамического
программирования на деревьях, изначально появившийся в контексте задачи выбора
набора инструкций для генерации машинного кода. Основой BURS является
регулярная древесная грамматика\cite{tata} со стоимостью применения подстановки,
то есть грамматика со следующим набором правил: 

$$
\begin{array}{rcll}
  N &:& \alpha& [c]\\
  N &:& \alpha\; (K_1,\dots,K_n)& [c]
\end{array}
$$

Здесь $N, K_i$ это нетерминалы, $\alpha$ --- терминал,
$c$ --- функция цены, описанная для каждого правила.
Как и для обычной линейной грамматики, вводится стартовый
нетерминал S. Считается, что терминальное дерево выводится в
данной грамматике, если его можно получить с помощью правил
подстановки из одноузлового дерева $S$.
Каждая подстановка заменяет нетерминал $N$, находящийся в листе дерева, на дерево 
$\alpha\;(K_1,\dots,K_n)$, если в грамматике есть правило
$N:\alpha\;(K_1,\dots,K_n)$. 
Для каждой подстановки вычисляется стоимость ее применения с помощью
функции цены ($c$).
Аргументами функции могут служить терминальная метка ($\alpha$) и стоимости
вывода поддеревьев.
Задачей, которую решает BURS, является поиск вывода наименьшей стоимости
для заданного дерева по заданной грамматике.
Такой вывод может быть найден двухпроходным алгоритмом.

Первый проход (\emph{пометка}) обрабатывает дерево снизу вверх и вычисляет
для каждого узла набор троек $(K,\;R,\;c)$, где $K$ --- нетерминал, из которого может быть
выведено поддерево с корнем в обрабатываемом узле,
$R$ --- первое правило, которое используется для вывода минимальной стоимости из $K$,
$c$ --- стоимость такого вывода.
Процесс пометки происходит следующим образом:

\begin{itemize}
\item для листовой вершины, помеченной терминалом $\alpha$, в множество троек
этого вершины добавляется $(K,\;R,\;c\:(\alpha))$ для каждого правила $R=K:\;\alpha\;[c]$;
\item для промежуточной вершины, помеченной терминалом $\alpha$,
с непосредственными поддеревьями $v_1,\dots,v_n$
в множество добавляется тройка $(K,\;R,\;c\:(\alpha,c_1,\dots,c_n))$ для каждого правила
$R=K:\;\alpha\;(K_1,\dots,K_n)\;[c]$, где $(K_i,\;R_i,\;c_i)$ входит в множество троек для
$v_i$; если есть несколько правил вывода из нетерминала $K$, то выбирается правило,
минимизирующее стоимость вывода.
\end{itemize}

Второй проход (\emph{свертка}) просматривает дерево сверху вниз, используя сделанные пометки.
Первое правило из минимального вывода определяется тройкой $(S,\;R,\;c)$ для корневого узла
(если такой тройки нет, то вывод из $S$ невозможен).
Это правило однозначно определяет нетерминалы $K_i$ для каждого непосредственного поддерева
и процесс повторяется.

Для пометки дерева потенциально необходимо каждое правило грамматики применить к каждому узлу.
При фиксированной грамматике алгоритмическая сложность первого прохода --- $O\:(|R|)$,
где $|R|$ --- количество правил
(размер множества троек для каждого узла ограничен числом нетерминалов, которое не больше,
чем количество правил). Свертка также имеет линейную сложность.


\subsection{Язык программирования Kotlin}

Язык Kotlin это функциональный, объектно-ориентированный язык,
разрабатываемый в компании JetBrains.
Kotlin был выбран для реализации принтер-плагина к IDEA по нескольким причинам. 
Во-первых, Kotlin обладает хорошей интеграцией с Java, что позволяет
использовать его без проблем с IDEA API. 
Во-вторых, функции в Kotlin являются объектами первого рода, что позволяет
легко реализовывать комбинаторные библиотеки на нем.
На данный момент Kotlin находится в стадии разработки, поэтому
периодически возникают
проблемы с тем, что исходные коды перестают быть совместимыми с новыми
версиями языка,
но обычно требуется внести небольшой набор исправлений для
восстановления работоспособности.
