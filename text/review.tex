\section{Обзор}

\subsection{Принтер-комбинаторы}
Все пустили корни из Oppen
Есть линейная реализация, и реализация On-line и много чего еще, но слабая выразительная мощь.

\subsection{Принтер-комбинаторы с выбором}
Достаточная выразительная сила, но существующая реализация экспоненциальна
относительно размеров входа.

\subsection{Средства форматирования кода в IDE}

\subsubsection{IDEA}
adas
\subsubsection{Eclipse}
sdasd

\subsection{BURS}
Bottom-up rewrite systems (BURS)~\cite{burs} --- это метод динамического
программирования на деревьях, изначально появившийся в контексте задачи выбора
набора инструкций для генерации машинного кода. Основой BURS является
регулярная древесная грамматика с весами, то есть грамматика со следющим
набором правил: 

$$
\begin{array}{rcll}
  N &:& \alpha& [c]\\
  N &:& \alpha\; (K_1,\dots,K_n)& [c]
\end{array}
$$

Здесь $N, K_i$ это нетерминалы, $\alpha$ --- терминал,
$c$ --- функция цены, описанная для каждого правила.
Similar to the ordinary \emph{``linear''} (or \emph{``word''}) grammar we distinguish certain starting 
nonterminal $S$ 
and say that terminal-labeled \emph{tree} is derivable in the given grammar if it can be constructed from 
a single node labeled by $S$ using repetitive substitutions. Each substitution replaces (arbitrary) leaf 
labeled by a nonterminal $N$ with a tree $\alpha\;(K_1,\dots,K_n)$ if there is a rule 
$N:\alpha\;(K_1,\dots,K_n)$ in the grammar. The cost functions are used to calculate the overall cost 
of a certain derivation. Each cost function can depend on the terminal label ($\alpha$) and the costs 
of the derivations for each subtree.

In the context of BURS we are interested in (arbitrary) least-cost derivation of a certain tree provided by 
a certain grammar. This derivation can be found by a two-pass algorithm. 

The first pass (\emph{labeling}) traverses the subject tree bottom-up and calculates for each its node
the set of all triplets $(K,\;R,\;c)$, where $K$ --- nonterminal from which the subtree rooted at the given node
can be derived, $R$ --- the first rule of the minimal derivation from $K$, $c$ --- the cost of this derivation.
The labeling process is performed as follows:

\begin{itemize}
\item for each leaf node labeled by a terminal $\alpha$ we add into the set for this node a 
triplet $(K,\;R,\;c\:(\alpha))$ for the each rule $R=K:\;\alpha\;[c]$;
\item for an intermediate node labeled by a terminal $\alpha$ with immediate successors $v_1,\dots,v_n$ we
add into the set for this node a triplet $(K,\;R,\;c\:(\alpha,c_1,\dots,c_n))$ for each rule 
$R=K:\;\alpha\;(K_1,\dots,K_n)\;[c]$ such that there is a triplet $(K_i,\;R_i,\;c_i)$ in the labeling for
the node $v_i$; if there are different suitable rules for the same nonterminal $K$ then we choose that 
which delivers the minimal cost.
\end{itemize}

The second pass (\emph{reduce}) is a top-down traversal which makes use of the constructed labeling. The first
rule of the minimal derivation is that from the triplet $(S,\;R,\;c)$ for the root node (if there is no such a
triplet, then there is no derivation from $S$). This rule unambiguously determines the nonterminals $K_i$ for 
each direct subtree of the root node and the process repeats.

To perform labeling we potentially need to try each rule of the grammar for each node of the tree; given 
the fixed grammar this results in $O\:(|R|)$ complexity, where $|R|$ is the number of rules  
(the size of the set of triples is limited by the number of the nonterminals which is in turn not 
greater then the number of rules). Reduce is linear as well.

\subsection{Язык программирования Kotlin}

Язык Kotlin это функциональный, объектно-ориентированный язык,
разрабатываемый в компании JetBrains.
Kotlin был выбран для реализации принтер-плагина к IDEA по нескольким причинам. 
Во-первых, Kotlin обладает хорошей интеграцией с Java, что позволяет
использовать его без проблем с IDEA API. 
Во-вторых, функции в Kotlin являются объектами первого рода, что позволяет
легко реализовывать комбинаторные библиотеки на нем.
На данный момент Kotlin находится в стадии разработки, поэтому
периодически возникают
проблемы с тем, что исходные коды перестают быть совместимыми с новыми
версиями языка,
но обычно требуется внести небольшой набор исправлений для
восстановления работоспособности.
