\section{Реализация}

\subsection{Библиотека принтер-комбинаторов на Haskell}
К сожалению, библиотека, описанная в \cite{swierstra}, обладает экспоненциальной
сложностью относительно входного документа. В рамках данной работы удалось реализовать
принтер-комбинаторную библиотеку (см. приложение~\ref{app:1}) с тем же набором комбинаторов,
которая линейна относительно входного документа и полиномиальна
относительно максимальной ширины вывода.

% В библиотеке из \cite{swierstra} варианты раскладок хранятся в списке, в начале которого
% располагаются лучшие представления. В момент, когда построен список для всей структуры,
% результирующий
% вариант выбирается простым взятием элемента из головы списка. Фактически элементы,
% которые обладают одинаковыми размерами в терминах структуры \lstinline[language=haskell]{Format},
% неотличимы на каждом шаге
% вычислений, а среди раскладок с одинаковыми ширинами выбор падет на ту,
% что обладает меньшей высотой.
% Поэтому логично ввести факторизацию представлений по полям
% \lstinline[language=haskell]{total_w} и \lstinline[language=haskell]{last_w}
% структуры \lstinline[language=haskell]{Format}.
% Такая факторизация ограничивает количество возможных представлений на каждом этапе вычислений

\subsubsection{Применение BURS}
Сведение задачи к BURS основано на следующих наблюдениях. Пусть $w$ есть максимальная допустимая
ширина вывода. Поскольку каждая возникающая раскладка представляется структурой
\lstinline[language=Haskell]{Format}, для поиска оптимальной раскладки можно рассматривать
все промежуточные как тройки $(t, l, h)$, где $t \le w$ --- это общая ширина раскладки, $l \le t$ ---
ширина ее последней строчки, $h$ --- ее высота. Тогда очевидно, что между $(t, l, h_1)$ и
$(t, l, h_2)$ первая тройка предпочтительней при $h_1 \le h_2$ на любом этапе вычислений.
Таким образом для каждого узла достаточно хранить $O(w^2)$ представлений.

\subsection{Библиотека принтер-комбинаторов на Kotlin}
Дополнительный комбинатор Fill

\subsection{Принтер-плагин Java для IntelliJ IDEA}

Все хорошо, но есть проблемы.

