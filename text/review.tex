\section{Обзор}

\subsection{Принтер-комбинаторы}

Большинство принтер-комбинаторных библиотек\cite{
swierstraChitil, swierstra04, peytonJones, kiselyov, chitil, swiComb}
являются развитиями библиотек
Джона Хьюза\cite{hughes} и Филиппа Вадлера\cite{wadler}, которые,
в свою очередь, представляют собой функциональную реализацию алгоритма
Оппена\cite{oppen}.

Среди упомянутых библиотек есть 


\subsection{Принтер-комбинаторы с выбором}
Достаточная выразительная сила, но существующая реализация экспоненциальна
относительно размеров входа.

\subsubsection{Структура Format}

\subsection{Средства форматирования кода в IDE}

\subsubsection{IDEA}
adas
\subsubsection{Eclipse}
sdasd

\input{burs}

\subsection{Язык программирования Kotlin}

Язык Kotlin это функциональный, объектно-ориентированный язык,
разрабатываемый в компании JetBrains.
Kotlin был выбран для реализации принтер-плагина к IDEA по нескольким причинам. 
Во-первых, Kotlin обладает хорошей интеграцией с Java, что позволяет
использовать его без проблем с IDEA API. 
Во-вторых, функции в Kotlin являются объектами первого рода, что позволяет
легко реализовывать комбинаторные библиотеки на нем.
На данный момент Kotlin находится в стадии разработки, поэтому
периодически возникают
проблемы с тем, что исходные коды перестают быть совместимыми с новыми
версиями языка,
но обычно требуется внести небольшой набор исправлений для
восстановления работоспособности.
