\section{Обзор существующих подходов}

\subsection{Принтер-комбинаторы}
Все пустили корни из Oppen

\subsubsection{Простые принтер-комбинаторы}
Есть линейная реализация, и реализация On-line и много чего еще, но слабая выразительная мощь.

\subsubsection{Принтер-комбинаторы с Choice}
Достаточная выразительная сила, но существующая реализация экспоненциальна
относительно размеров входа.

\subsection{BURS}
Здесь будет определение BURS, из какой задачи оно появилось и т.д.

\subsection{Язык программирования Kotlin}

Язык Kotlin это функциональный, объектно-ориентированный язык, разрабатываемый в компании JetBrains.
Kotlin был выбран для реализации принтер-плагина к IDEA по нескольким причинам.  
Во-первых, Kotlin обладает хорошей интеграцией с Java, что позволяет использовать
его без проблем с IDEA API. 
Во-вторых, функции в Kotlin являются объектами первого рода, что позволяет
легко реализовывать комбинаторные библиотеки на нем.
На данный момент Kotlin находится в стадии разработки, поэтому периодически возникают
проблемы с тем, что исходные коды перестают быть совместимыми с новыми версиями языка,
но обычно требуется внести небольшой набор исправлений для восстановления совместимости.

\subsection{Средства форматирования кода в IDE}

\subsubsection{IDEA}
adas
\subsubsection{Eclipse}
sdasd
