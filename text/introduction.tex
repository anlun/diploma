\section*{Введение}

В жизненном цикле программного обеспечения важную роль занимает этап
поддержки [здесь ссылка на статью De Marco?]. Этот этап по разным оценкам
составляет от 30 до 60\% общего бюджета проекта. Поэтому для
промышленного программного кода важно, чтобы его было легко читать и изменять.
Рассмотрим описание функций
на рисунках~\ref{fig:wikiExUnfor} и \ref{fig:wikiExBSD}.

\begin{figure}[h!]
	\centering
	\lstinputlisting[language=C]{codes/wikiExUnfor.c}
	\caption{Неформатированный код}
	\label{fig:wikiExUnfor}
\end{figure}

\begin{figure}[h!]
	\centering
	\lstinputlisting[language=C]{codes/wikiExBSD.c}
  \caption{Форматированный код (BSD)}
	\label{fig:wikiExBSD}
\end{figure}

Эти функции семантически и синтаксически эквиваленты с точки зрения
компилятора C, но вариант с рисунка~\ref{fig:wikiExUnfor} гораздо хуже
поддается понимаю человека и требует большего времени на изменение,
что увеличивает стоимость и трудозатраты на поддержку. Как мы видим из этого
примера, программные тексты должны явным образом отражать
структуру синтаксического дерева программы.

Кроме того, распространенной практикой [ссылка на статью про стандарты
кодирования], доказавшей свою состоятельность, является использования
общепроектного стандарта кодирования (\textit{coding convention}).
\textbf{Стандарт кодирования (СК)} ---
это набор соглашений, которые используются
при написании программного кода. В него входят: способы выбора имен переменных
и других идентификаторов, стили отступов при оформлении логических блоков,
способы расстановки ограничителей логических блоков (скобок),
форматы комментариев. СК также призван упростить анализ и изменение
программы, поэтому его важно соблюдать, что вводит
дополнительные ограничения на исходные тексты. СК разных проектов, даже
реализуемых на одном и том же языке программирования, могут существенно
различаться.
Так код на рисунке~\ref{fig:wikiExBSD} соответствует СК BSD,
а на рисунке~\ref{fig:wikiExGNU} --- GNU.

\begin{figure}[h!]
	\centering
	\lstinputlisting[language=C]{codes/wikiExGNU.c}
  \caption{Форматированный код (GNU)}
	\label{fig:wikiExGNU}
\end{figure}

Кроме ситуаций, когда СК в проекте поддерживается при ручном написании
программного кода, существует другой важный пример использования СК ---
в языковых процессорах. \textbf{Языковой процессор} (\textbf{ЯП}) ---
это программное средство, принимающее на вход программу в виде текста
на некотором языке (программирования, разметки и т. д.) и решающее
определенную задачу над этой программой. К языковым процессорам можно
отнести: компиляторы, суперкомпиляторы, интерпретаторы,
средства статического анализа кода, декомпиляторы, средства рефакторинга,
средства реинжиниринга, интегрированные среды разработки (IDE) и др.


