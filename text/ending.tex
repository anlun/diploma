\section*{Заключение}

\subsection*{Результаты}
\addcontentsline{toc}{subsection}{Результаты}

В рамках данной работы достигнуты следующие результаты:
\begin{enumerate}
\item Для задачи поиска оптимальной раскладки документа,
определенного с помощью комбинаторов из \cite{swierstra}, на заданную ширину
конструктивно доказана ее линейность относительно размеров дерева комбинаторов;
\item Написана соответствующая статья\cite{podkopaevBoulytchev},
которая была принята на конференцию PSI'14\footnote{
PSI, \cd{http://psi.nsc.ru/}};
\item Разработан подход задания принтеров с помощью образцов для языков программирования;
\item Реализован плагин форматирования программных текстов на языке Java для IntelliJ IDEA
в рамках апробации общего подхода.
\end{enumerate}


\subsection*{Дальнейшая работа}
\addcontentsline{toc}{subsection}{Дальнейшая работа}

Существует несколько направлений для развития данной работы.
В рамках плагина стоит добавить
анализ эталонного репозитория на полноту, чтобы для полученного по
эталонному коду принтера можно было гарантировать возможность
обработки любого абстрактного синтаксического дерева Java.
Также для возможности практического применения необходимо повысить
производительность плагина.
Для общего подхода самой важной задачей является разработка
абстрагированной от целевого языка программной системы. Кроме того,
необходимо получить способ описания порядка поддеревьев для тех
случаев, когда он не задан явно синтаксисом языка.
