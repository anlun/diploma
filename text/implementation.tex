\section{Реализация}

\subsection{Сведение Pretty-Printing к BURS}
К сожалению, библиотека, описанная в \cite{swierstra}, экспоненциальна
относительно входного документа. Для снижения алгоритмической сложности
в рамках данной работы произведено сведение задачи к BURS.

Сведение основано на следующих наблюдениях. Пусть $w$ есть максимальная допустимая
ширина вывода. Поскольку каждая возникающая раскладка представляется блоком
\lstinline[language=Haskell]{Format}, для поиска оптимального представления можно рассматривать
все промежуточные раскладки
как тройки $(n, k, h)$, где $n \le w$ --- это общая ширина раскладки, $k \le n$ ---
ширина ее последней строчки, $h$ --- ее высота. Тогда очевидно, что при $h_1 \le h_2$
между $(n, k, h_1)$ и $(n, k, h_2)$ первая тройка предпочтительней на любом этапе вычислений.
Таким образом для каждого узла не может быть более $w^2$ существенных представлений\footnote{
На самом деле, из-за ограничения $k \le n$ максимальное число раскладок не $w^2$, a
$\frac{w^2 + w}{2}$, но это несущественно.}.

Документ, для которого мы ищем раскладку, можно рассматривать как дерево, построенное из примитивов
\lstinline[language=Haskell]{text},
\lstinline[language=Haskell]{indent},
\lstinline[language=Haskell]{beside},
\lstinline[language=Haskell]{above} и \lstinline[language=Haskell]{choice}. Тогда задачу
раскладки документа можно решать независимо для поддеревьев, а потом объединять решения
при учете, что для каждого поддерева
и каждой пары $(n, k), k \le n \le w$ запоминается минимальная $h$ такая, что для поддерева существует
текстовое представление с размерами $(n, k, h)$. Тогда, совершив обход дерева снизу вверх, мы сможем
вычислить оптимальное представление для всего дерева. Полезно заметить,
что оптимальное представление дерева не всегда получается из оптимальных представлений его
поддеревьев, поскольку при использовании поддеревьев возникают дополнительные
ограничения на ширину раскладок этих поддеревьев.

Приведенные наблюдения можно формализовать в виде BURS-задачи.
Для заданной ширины вывода $w$ введем набор нетерминалов
$T_n^k$, для всех $k \le n \le w$. Определим BURS-грамматику так, чтобы цена вывода $h$ нетерминала
$T_n^k$ для документа соответствовала его раскладке с параметрами $(n, k, h)$.
Для такой BURS-грамматики этап разметки посчитает все существенные раскладки, а этап свертки
вернет оптимальную. Определим правила переписывания для этой грамматики:
\begin{enumerate}
\item Для терминального узла $[\mbox{\lstinline{text s}}]$\footnote{
  Квадратные скобки используются для обозначения терминалов, состоящих из одного
  или нескольких символов.}
  существует два варианта:
  \begin{itemize}
     \item Если $|s|\le w$ (где $|s|$ --- это длина строки $s$), то вводится единственное правило
           $T^{|s|}_{|s|}: [\mbox{\lstinline{text s}}]$ со стоимостью $1$;
           
           для всех остальных
           $k, n\ne |s|$ используется $T^k_n:[\mbox{\lstinline{text s}}]$ со стоимостью $\infty$;
     \item Если $|s| > w$, то используется $T^k_n:[\mbox{\lstinline{text s}}]$ со стоимостью\
       $\infty$ для всех $k, n$.
  \end{itemize}
  Действительно, раскладка документа, состоящего из строчки длины $|s|$, может быть только размеров
  $(|s|, |s|, 1)$. Все остальные размеры недоступны, поэтому имеют стоимости $\infty$.

\item Для узла $[\mbox{\lstinline{indent m}}]$ введем два набора правил:
  \begin{enumerate}
     \item $T^{k+m}_{n+m}:[\mbox{\lstinline{indent m}}]\:(T^k_n)$ со стоимостью,
       равной стоимости раскладки поддерева $m$ в $T_n^k$, для всех $n$ 
     и $k$ таких, что $n+m\le w$ и $k\le n$;
     \item $T^k_n:[\mbox{\lstinline{indent m}}]\:(T^i_j)$ со стоимостью $\infty$ в противном случае.
  \end{enumerate}
  Понятно, что сдвиг раскладки с параметрами $n$, $k$ и $h$ на $m$ позиций вправо
  создает раскладку с параметрами $n+m$, $k+m$, $h$. Такая раскладка допустима, если
  $n+m\le w$ и $k+m\le w$.

\item Для узла $[\mbox{\lstinline{above}}]$ вводим правило $T^{k_2}_{\max(n_1, n_2)}:[\mbox{\lstinline{above}}]\:(T^{k_1}_{n_1},\;T^{k_2}_{n_2})$ 
со стоимостью, равной сумме стоимостей вывода поддеревьев,
для всех $k_1\le n_1\le w$ и $k_2 \le n_2 \le w$.

Действительно, при вертикальном соединении раскладок с параметрами
$n_1$, $k_1$, $h_1$ и $n_2$, $k_2$, $h_2$ мы получаем раскладку с размерами
$\max\:(n_1,n_2)$, $k_2$, $h_1+h_2$. Вертикальная композиция допустимых раскладок всегда допустима.

\item Для узла $[\mbox{\lstinline{beside}}]$ вводим правило
  $T^{k_1+k_2}_{\max\:(n_1,\:k_1+n_2)}:[\mbox{\lstinline{beside}}]\:(T^{k_1}_{n_1},\:T^{k_2}_{n_2})$
  для каждой комбинации $n_1, n_2, k_1, k_2$ такой, что $k_1+k_2\le\max\:(n_1,\:k_1+n_2)\le w$.
  Стоимость такого вывода равна сумме стоимостей вывода поддеревьев минус 1.
  Это может быть легко проверено из геометрических соображений.

\item Для узла $[\mbox{\lstinline{choice}}]$ вводим правило
  $T^k_n:[\mbox{\lstinline{choice}}]\:(T^k_n,\:T^k_n)$ для всех $k\le n\le w$.

  Цена есть минимум среди цен вывода для поддеревьев. Понятно, что среди двух раскладок
  с одинаковыми ширинами выбирается та, что имеет меньшую высоту.
\end{enumerate}

Для завершения описания грамматики необходимо ввести правило для стартового нетерминала $S$.
Можно добавить правило $S : r$ с тождественной функцией цены для любого $r$, или ввести
цепное правило $S: T_n^k$ с такой же функцией для всех нетерминалов $T_n^k$
(такое правило требует несущественного расширения определения BURS).

Число нетерминалов в определенной выше грамматике есть $O(w^2)$.
Однако, число правил есть $O(w^4)$, так
как в дереве есть узлы со степенью 2. Так, приведенная BURS реализация оптимального принтера
работает за линейное время от числа узлов в дереве документа для фиксированной ширины вывода, при
этом сложность от ширины вывода есть $O(w^4)$. Понятно, что данное сведение может быть выполнено и
для других примитивов построения дерева, которые могут иметь большую степень, но ценой
экспоненциального роста сложности по $w$.


\newpage
\subsection{Библиотека принтер-комбинаторов на Haskell}

В рамках данной работы удалось реализовать
принтер-комбинаторную библиотеку (см. \cite{haskellImpl}) с тем же набором комбинаторов,
что и в~\cite{swierstra},
которая линейна относительно входного документа и полиномиальна
относительно максимальной ширины вывода за счет сведения к BURS, описанного выше.
Данная реализация использует низкоуровневые типы и функции из оригинальной
библиотеки~\cite{swierstra}, в то время как высокоуровневые типы и комбинаторы переопределены.

В реализации нет непосредственного использования BURS: в явном виде не задается BURS-грамматика,
нет набора нетерминалов и стандартного алгоритма обработки.
Вместо этого для каждого узла документа вычисляется ассоциативный массив, который
ставит в соответствие паре $(n, k)$ лучший (минимальный по числу строк) Format с параметрами
ширин $n$ и $k$. Так, в ассоциативном массиве записаны пары $((n, k), f)$, где $f$ --- это
блок Format, соответствующий оптимальному представлению с ширинами $(n, k)$. Значение
высоты $f$ есть минимальная стоимость вывода нетерминала $T^k_n$ для данного узла.
Заметим, что поскольку $f$ является конечным результатом для $T^k_n$, а не просто
последним правилом, которое нужно применить для построения ответа, то отменяется
необходимость применения в дальнейшем алгоритма свертки.
В конце, для корня дерева, представляющего документ, выбирается элемент минимальной стоимости из
ассоциативного массива.

Так как наибольший интерес в рамках данного исследования представляет поведение библиотеки
в худшем случае, реализация была апробирована на искусственно-сгенерированных документах.
Каждый из них представляет собой дерево, построенное с помощью базовых комбинаторов.

TODO: описать построение документов

Результаты времени исполнения для сравнения авторской реализации с оригинальной библиотекой
приведены в таблице~\ref{tbl:time}.
Из приведенных данных видно, что, начиная с некоторой комбинации ширины вывода и размера документа,
библиотека~\cite{swierstra} не может вычислить раскладку документа.
Значения, записанные подобно \mbox{``-($>$ 59)''},
указывают, что после указанного времени процесс аварийно завершился с переполнением стека.

Авторская реализация часто не показывает линейное поведение, как это ожидается при фактическом
учетверении числа узлов в документе от строчки к строчке. Дополнительные эксперименты показали
связь данного поведения с разреженностью ассоциативного массива для больших ширин.
Другими словами, для маленьких деревьев число значений в соответствующем ассоциативном массиве
сильно меньше верхней границы. С ростом размера дерева число записей растет
экспоненциально, пока не достигнется верхняя граница.

Кроме того, особенностью описанной реализации является то, что она не использует 
ленивые вычисления для ускорения построения оптимального текстового представления для
поданного на вход документа, а, наоборот,
форсирует большинство вычислений, что делает ее легко портируемой на языки
без встроенной ленивости.

\input{haskellTbl}

\newpage
\subsection{Библиотека принтер-комбинаторов на Kotlin}

\subsubsection{Мемоизация вычислений для поддеревьев}

В реальных принтерах документы, состоящие из комбинаторов
\lstinline[language=Haskell]{text},
\lstinline[language=Haskell]{indent},
\lstinline[language=Haskell]{beside},
\lstinline[language=Haskell]{above} и \lstinline[language=Haskell]{choice}, чаще всего
представляют собой не дерево, а \textit{дэг} (DAG, Direct Acyclic Graph). Поэтому при наивной
реализации BURS можно получить экспоненциальную сложность от размеров документа,
поскольку соответствующее дерево будет иметь экспоненциальный размер от размера дэга.
Чтобы избежать этой проблемы достаточно завести HashMap, который будет хранить соответствие
поддерева с посчитанным для него набором раскладок. Опять же в наивной реализации,
это сделает алгоритм квадратичным от размера дэга, так как вычисления хэша для дерева занимает
линейное время. Чтобы преодолеть и эту сложность, надо в дереве хранить вычисленный хэш.
К сожалению, это сложно выразить на Haskell, поэтому эта оптимизация оставлена только для реализации
на Kotlin.

\subsubsection{Комбинатор Fill}

Для реализации принтера, использующего шаблоны, необходимо добавить
дополнительный принтер-комбинатор в бибилиотеку --- комбинатор Fill.

\begin{figure}[h!]
  \centering
	\subfloat[]{
		\centering
    \tikz[scale = 2.5]{
       \draw (0,0) -- (1,0) -- (1,0.2) -- (2,0.2) -- (2,1) --
          (0, 1) -- cycle;
       \draw (0,-0.1) -- (0,-1.1) -- (1.1,-1.1) -- (1.1,-0.9) --
          (2.3,-0.9) -- (2.3, 0.1) -- (1.1,0.1) -- (1.1,-0.1) -- cycle;
    }
		\label{fig:fill1}
	}
	\quad
  \subfloat[][fc --- константа сдвига]{
		\centering
    \tikz[scale = 2.5]{
          \draw (0,0) -- (1,0) -- (1,0.2) -- (2,0.2) -- (2,1)
            -- (0, 1) -- cycle;
          \draw (0.3,-0.1) -- (0.3,-1.1) -- (1.1,-1.1) -- (1.1,-0.9)
            -- (2.3,-0.9) -- (2.3, 0.1) -- (1.1,0.1) -- (1.1,-0.1) -- cycle;
          \draw[<->] (0,-0.6) -- (0.3,-0.6);
          \draw (0.15,-0.4) node [fill=white] {\tiny fc};
          \draw[dashed] (0,0) -- (0,-1.1);
       }
		\label{fig:fill2}
	}
	\caption{Оператор Fill}
\end{figure}

TODO: Примеры, зачем нужно. К примеру, PsiCodeBlock

К сожалению, теперь становится недостаточно факторизовать структуры Format
только по общей ширине и ширине последней строчки.
С комбинатором Fill фактор-пространство дополнительно параметризуется и первой
строчкой.

TODO: привести примеры

\subsubsection{Дополнительная фильтрация вариантов}

Для того, чтобы избежать экспоненциальной сложности вычисления оптимальной
раскладки документа, мы ввели факторизацию по размерам (ширинам).
Количество раскладок для каждого поддерева теперь ограничено, но, к сожалению,
слишком большой константой --- $O(w^3)$ ($O(w^2)$ для набора комбинаторов без
Fill), а сложность вычисления узлов Beside, Above, Fill --- $O(w^6)$
($O(w^4)$ без комбинатора Fill). Факторизация была основана на том факте, что
среди раскладок с одинаковыми ширинами в любом дальнейшем использовании
оптимальнее вариант с меньшей высотой. Для дополнительной фильтрации множества
вариантов воспользуемся рассуждением, что если раскладка $A$ обладает и
меньшими ширинами, и меньшей высотой, чем раскладка $B$, то, как и раньше,
достаточно оставить только ее во множестве вариантов.

Таким образом мы ввели отношение частичного порядка на раскладках, а
описанная фильтрация --- поиск минимумов на множестве представлений\cite{poset}.
Наивная реализация поиска имеет квадратичную сложность от размера множества,
таким образом асимптотика алгоритма не ухудшается из-за этих дополнительных
действий. К сожалению, данная оптимизация и не улучшает поведение
в худшем случае, так как
можно привести пример, когда все множество раскладок с допустимыми ширинами
будет состоять из несравнимых элементов, но на практике количество вариантов
существенно сокращается, что делает весь подход применимым на реальных данных.

\newpage
\subsection{Принтер-плагин Java для IntelliJ IDEA}

TODO

Стоит заметить, что с инженерной точки зрения для решения отдельно стоящей задачи
форматирования по образцу кода на Java в IntelliJ IDEA лучше подходит
получение настроек для существующего форматтера с помощью известных техник обучения.
Подобный подход используется в работе \cite{learning}.
Но наш метод позволяет выразить представления, которые не предусмотрены
встроенным форматтером, кроме того потенциально предоставляет возможность проще
создавать форматтеры для новых языков. 

\subsubsection{Получение шаблонов}
Тут надо написать, как же таки получаются шаблоны.

Получаются из эталонного репозитория.

\subsubsection{Анализ производительности}

Для оценки производительности были рассмотрены самые большие исходные файлы
проекта IntelliJ IDEA Community Edition\footnote{
IntelliJ IDEA Community Edition,
http://github.com/jetbrains/intellij-community/}.
Результаты производительности приведены в
таблице~\ref{tbl:pluginPerformanceTbl}.

\begin{table}[h!]
	\centering

  \subfloat[Ширина 200] {
    \centering
    \begin{tabular}{c c c}
    Имя файла & Количество строк & Время \\
    \hline
    Messages.java & 2007 & 1.76 \\
    UIUtil.java & 2808 & 1.56 \\
    AbstractTreeUi.java & 5112 & 2.25 \\
    EditorImpl.java & 6789 & 2.74 \\
    ConcurrentHashMap.java & 7191 & 2.05 
    \end{tabular}
  }

  \subfloat[Ширина 250] {
    \centering
    \begin{tabular}{c c c}
    Имя файла & Количество строк & Время \\
    \hline
    Messages.java & 2007 & 1.01 \\
    UIUtil.java & 2808 & 1.58 \\
    AbstractTreeUi.java & 5112 & 2.68 \\
    EditorImpl.java & 6789 & 2.93 \\
    ConcurrentHashMap.java & 7191 & 2.03 \\
    \end{tabular}
  }

	\caption{Время форматирования файлов (в секундах)}
	\label{tbl:pluginPerformanceTbl}
\end{table}

\subsubsection{Открытые проблемы}

К сожалению, есть несколько принципиально нерешенных проблем. Первой такой
проблемой является печать синтаксических структур с переменным числом
подвыражений. К примеру, это списки и бинарные выражения. Проблема вызвана
тем, что непонятно, как для таких структур задавать шаблоны. Можно хранить
представления для списков по количеству элементов
до некоторой максимальной длины, но,
во-первых, это достаточно обременительно, учитывая количество разнообразных
типов списков, а, во-вторых, такой подход не сможет работать на списках большей
длины. В рамках данной работы было реализовано два разных решения для данной
задачи --- классический для форматтеров из IDE и с использованием шаблона для
пары элементов списка.

Классический способ решения проблемы для списков заключается в том, что
списки просто печатаются заполняющим методом (см. пример на
рисунке~\ref{fig:listClassicEx}).

\begin{figure}[h!]
  \textbf{Тут надо привести какой-нибудь пример}

  \caption{Пример классической печати списков}
  \label{fig:listClassicEx}
\end{figure}


Второй проблемой является задание порядка поддеревьев для конструкций, которые
его явным образом не определяют. К примеру, это верно для классов в Java.
